\section{Introduction \label{sec:intro}}

%\subsection{Introduction}
The last financial crisis and more recently the dramatic events surrounding Greece has triggered a re-design of portfolio strategies among practioners. Uncertainty about future asset returns within a portfolio optimization framework has led the financial industry to look for new solutions to propose to their clients.\\

In   portfolio management practice, major  difficulties originate from problems associated with reliable  estimation of statistical model parameters and the sensitivity of the optimal asset allocation with respect to these quantities. 
Since there are several aspects of these difficulties and an entire range of methods which  address them, let us explain our concept focusing on classical dynamic portfolio optimization. \\

In this context, the major quantitative ingredients are the (conditional) means and the covariances  of the so-called asset log-returns.  While a reliable and statistically significant estimation of log-return means is virtually impossible,  the estimation of covariances may also be extremely difficult in practice, since for a large asset number, the consideration  must take into account the  asymptotic behavior of the spectrum of random empirical covariance matrices. Given permanent time changes of  the price fluctuation intensity and an extreme sensitivity of the optimal portfolio weights, a naive construction of the  optimal portfolio is usually  has no value for practical applications.\\
 
In view of these problems, some practically relevant  approaches to  portfolio optimization have been suggested in order to overcome or to  diminish the dependence on statistical procedures of  model identification. Let us emphasize the  {\it benchmark approach} in this setting. This theory addresses an  optimal portfolio selection  under minimal theoretical assumptions and presents considerations, justifying the asymptotic optimality of an equally-weighted portfolio. This investment strategy attempts to hold an approximately equal fraction  of the entire portfolio wealth in each of the assets, selected for the investment. 
Since  this strategy requires a regular position re-balancing, it is  not a static investment, strictly speaking. However, empirical investigations show that for appropriate diversification, even infrequent re-balancing can achieve a  reasonable performance.\\
 
A similar area of ideas is related to the so-called {\it risk-parity} approach. In this framework, the investor attempts to build a portfolio choosing  portfolio weights such that the marginal contribution from each asset position  to an appropriately defined {\it total  portfolio risk} is the same.
Such {\it risk parity} approach is used to build diversified portfolios which do not rely on a return expectations, with the focus on risk management rather than performance. However, the  risk parity approach has also been  criticized, and some stylized dependence on expected returns has been  reintroduced, with extensions in terms of the so-called {\it minimum-torsion} approach.\\
 
A general framework is suggested in \cite{Platen2009}, the so-called {\it benchmark approach} which assumes the existence of a numeraire portfolio. This numeraire portfolio displays positive weights and when used as a benchmark renders all benchmarked portfolios to super-martingales. Platen shows that this portfolio is equivalent to the Kelly portfolio which maximizes a logarithmic utility function. This numeraire portfolio cannot be systematically outperformed by any other long-only portfolio. This theoretical numeraire portfolio can be approximated by a worldwide diversified portfolio. 
 
 






